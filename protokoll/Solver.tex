\subsection{Forward Euler}
\subsubsection{theoretical background}
\begin{align}
    y_{k+1}=y_k+h\cdot f(t_k, yk)
\end{align}
\subsubsection{Matlab Code}
\lstinputlisting[language=Matlab, caption=Matlab function forward Euler method]{forward_euler.m}

%%%%%%%%%%%%%%%%%%%%%%%%%%%%%%%%%%%%%%%%%%%%%%%%%%%%%%%%%%
\subsection{Runge-Kutta second order method (Midpoint)}
\subsubsection{Matlab Code}
\lstinputlisting[language=Matlab, caption=Matlab Function runge kutta method]{runge_kutta_so.m}

%%%%%%%%%%%%%%%%%%%%%%%%%%%%%%%%%%%%%%%%%%%%%%%%%%%%%%%%%%
\section{overall Program structure}
\lstinputlisting[language=Matlab, firstline=16, lastline=34, caption=Testing method for the different solver, label=PrintingFunction]{test_ode1.m}
As seen in listing \ref{PrintingFunction} is every Function tested with the three different solving methods tested. In addition to that, the ideal function is printed with a greatly increased accuracy. This ensures that the best-possible comparison can be pulled from this plots. 